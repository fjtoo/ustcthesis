% !TeX root = ../main.tex

\chapter{结论与展望}

\section{结论}

为了提升企业效能,贯彻敏捷开发思想,同时解决以Jenkins为代表的市面主流CI/CD流水线平台的痛点,
本文通过调研了国内外CI/CD流水线系统的研究现状,基于对企业内不同团队人员的需求分析,经过系统的概要设计、详细设计与测试,
完成了基于容器化的CI/CD流水线调度系统的设计与开发。
系统能够满足用户方便地对流水线以及流水线作业的构建与设计,允许用户进行丰富的人工干预操作,
同时允许用户管理流水线作业所使用的镜像与执行节点,并在流水线运行过程中对作业进行调度,控制其中各个实体的流转状态,最终将作业分发个不同的作业执行器进行执行,得到执行结果与产物。
经过功能与性能测试,能够满足用户需求。具体来说,本文完成的工作主要有以下几点。

首先,搜集并分析了当前CI/CD流水线系统的研究和发展现状,发现了以Jenkins为代表的主流系统的不足之处,明确了本系统着重需要实现的目标。
随后简要地介绍了Docker容器、Kubernetes和RocketMQ消息队列等技术,分析了其与CI/CD系统的相性与关联。

然后,对系统进行了概要设计与详细设计。系统将整个服务拆分为Backend、调度器与执行器三个服务,分别介绍了各个服务的设计思路与服务之间的交互,
并且从流水线管理、节点管理与镜像管理三个功能模块出发,具体地介绍了其内部算法与数据结构设计,并最终完成了基于CI/CD流水线调度系统的设计与开发。

最后,对系统的功能与性能进行测试。经过对测试用例的验证,以及对响应时间和作业执行时间的实验,证明本系统能够满足用户地使用需求。

\section{工作展望}

本系统为基于容器化的CI/CD流水线系统建设提供一种可行的解决方案,但系统中仍存在一些值得持续优化的方面,总结如下:

第一,执行器与调度器之间尚不具备心跳上报。如果一个正在运行作业的节点宕机,目前还无法立即恢复节点上正在运行的作业,使得用户无感知,这是一个将来重点优化的方向。

第二,从易用性的角度,对流水线的配置应更加模板化、插件化。
虽然本系统已经能够以很友好的方式让用户自己创建流水线、阶段、作业和插件,但是如果能够预设流水线模板、作业模板共用户在构建时进行选择,用户的使用效率和使用体验将有进一步提升,
同时可以考虑将任务以插件化的形式进行封装,提供插件时长共用户使用。后续可以考虑开发模板和插件的功能加入到系统中。

第三,节点负载情况对用户不透明。目前用户创建节点后,节点内部的作业运行、负载率等指标用户是无法看到的,使得用户对节点的管理失去了参考。
为提高用户进行节点管理的体验,应考虑对节点中部署监控工具或脚本,监控节点的CPU、内存、磁盘空间和网络使用情况,以监控大盘的形式提供给用户或管理员。

第四,系统存在性能瓶颈。
通过第六章的测试我们发现,当每分钟作业量超过35个时,系统开始出现了性能下滑,为了保证系统在未来能够承载更多业务,应对执行器硬件配置进行升级或增加服务器数量。