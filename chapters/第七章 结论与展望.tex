% !TeX root = ../main.tex

\chapter{结论与展望}

\section{结论}

为了提升企业效能,贯彻敏捷开发思想,同时解决以Jenkins为代表的市面主流CI/CD流水线平台的痛点,
本文通过调研了国内外CI/CD流水线系统的研究现状,基于对企业内不同团队人员的需求分析,经过系统的概要设计、详细设计与测试,
完成了基于容器化的CI/CD流水线调度系统的设计与开发。
系统能够满足用户对流水线以及流水线作业的构建与设计,允许用户进行丰富的人工干预操作,
同时允许用户管理流水线作业所使用的镜像与执行节点,并在流水线运行过程中对作业进行调度,控制其中各个实体的流转状态,最终将作业分发个不同的作业执行器进行执行,得到执行结果与产物。

具体来说,本文完成的工作主要有以下几点:



\section{工作展望}

目前为止,国内外对于CI/CD平台建设还未形成一套通用的最佳实践

首先,