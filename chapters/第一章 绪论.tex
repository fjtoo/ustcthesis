% !TeX root = ../main.tex

\chapter{绪论}

\section{系统开发背景和意义}
随着软件行业的快速发展,软件产品逐渐产生了需求变更频繁、产品迭代加快的趋势,这对软件的开发模式提出了新的挑战。
在过去,传统的软件开发团队普遍遵循瀑布模型,这一模型强调阶段划分,各个阶段必须严格按照时间先后进行,而且每个阶段都必须有产出物才能开始下个阶段\cite{瀑布模型}。
但是由于瀑布模型需要详尽的文档,这就要求在软件开发之初就对用户需求有着全面的了解,这在当下需求频繁变更的互联网时代几乎不可能;
同时,由于瀑布模型的集成式、阶段式的特点,产品的更迭速度势必缓慢,这也不符合现代的快速产品更迭的需求。
所以在当下,瀑布模型已经不符合大部分企业的要求,于是敏捷开发、DevOps、持续集成、持续交付等概念应运而生。

敏捷开发模型强调一组开发流程的反复迭代,要求开发人员采用XP、SCRUM等方法,通过紧密协作实现快速开发和交付。
敏捷开发鼓励以开发人员之间的频繁交流取代严格的文档,同时强调快速发布出软件的初代原型,以满足用户的基本需求,其余的功能和优化点将在后续的反复迭代中逐步完善\cite{敏捷开发}。
敏捷开发比瀑布模型更为灵活,但同时在开发人员和运维人员之间产生了矛盾:
开发人员追求快速迭代交付,而频繁的版本交付势必会加剧软件风险,与运维人员追求系统稳定形成了矛盾。

在开发与运维人员矛盾不断增加的背景下,DevOps(Development和Operations)理念产生并开始成为主流。
DevOps理念强调开发与运维的联合,旨在促进开发和运维过程中信息的流通,打破不同团队间的隔阂,以应对软件的快速变化\cite{DevOps}。
它要求运维人员不应等到软件交付后再进行运维,而是参与进软件的开发初期阶段,与开发人员的迭代周期进行同步的集成与交付。
但是,高频的测试、部署与交付显然不能依赖人力进行操作,这就要求DevOps拥有一套能够持续集成和持续交付的自动化工具。

持续集成(Continuous Integration,CI)和持续交付(Continuous Delivery,CD)是DevOps理念中重要的方法论。
持续集成是软件开发的迭代过程,团队成员不断上传代码至代码仓库,进行集成和测试,实现软件的快速迭代\cite{绪论持续集成1}。
这种方法不仅实现了DevOps的理念,及早发现问题和在早期交付可用原型,还通过强化开发团队与运维团队的联系,降低了团队成员的人工任务。
团队成员在持续集成中频繁集成工作成果,每日至少一次,通过自动构建的检验迅速发现集成错误。
与此同时,持续交付的核心思想是通过创建可重复、可靠的操作过程(一般称为流水线),将软件从概念变为现实\cite{绪论持续集成2}。
持续交付流水线将软件交付过程分成多个阶段,每个阶段验证新特性的质量,确认新功能,防止潜在错误对用户造成不良影响。
持续集成与持续交付继承了DevOps开发理念,成为现代软件工程中的必要环节,这种综合应用不仅简化了开发流程,也提高了软件开发的效率和质量。

当前,各个企业内均普遍使用持续集成与持续交付流水线(下文称“CI/CD流水线”)服务来完成内部的系统集成与交付。
市面上以Jenkins为代表的CI/CD流水线服务可以满足企业的基本开发需求,但随着企业业务的拓展与系统数量的增加,已经越来越难以满足不同团队的个性化需求。

结合以上背景,设计并且研发一套能够满足不同类型开发团队需求,且能保证高易用性、高稳定性的CI/CD流水线系统是非常有必要的。


\section{国内外研究现状}
CI/CD的概念,最早由ThoughtWorks公司的Martin Fowler系统性的阐述及推广\cite{CI首作},同时ThoughtWorks公司也开发第一个真正意义的持续集成工具:
Cruise Control\cite{绪论持续集成1},自此以后,企业和软件团队逐渐意识到了持续集成的价值和必要性。

与此同时,容器技术以及容器集群管理技术的高速发展,为CI/CD的自动化部署提供了很好的载体\cite{docker},也使得容器化成为CI/CD系统的主要特征。
Docker作为轻量级的容器解决方案,它可以快速启动,快速部署,容易移植,天生就适合做持续交付。
而谷歌的Kubernetes作为大规模容器集群编排管理引擎,支持自动化部署、管理大规模可伸缩的容器集群,也促进了CI/CD的发展。
这些技术使得CI/CD工具如雨后春笋一般涌现出来,如 Jenkins、ThoughtWorks Go CD、Codeship等,它们都旨在解决应用持续交付的问题,简化应用的开发和部署过程。
在工业界,各大云服务商以及公司也逐渐选择Kubernetes与Docker作为构建持续交付平台的技术体系。




国内的CI/CD系统主要由一些大型互联网企业和云服务商打造,以满足企业内部软件团队的需要,提高企业研发效能,同时也向外部用户提供服务。
如腾讯的蓝盾平台、华为的DevCloud平台、阿里巴巴的云效平台等等。
这些平台往往基于企业自身在互联网行业的一系列成功实践,以Web的形式为用户提供可视化的友好界面,对内承载企业自有业务的CI/CD任务,对外也可以满足不同类型客户的多元化需求。

现如今,CI/CD流水线平台得到了极为广泛的应用。据调查,有六成的企业和软件团队使用Jenkins作为CI/CD平台\cite{DevOps中国调查研究};另一部分,




\section{解决的主要问题}
本系统致力于解决现有以Jenkins为代表的主流CI/CD流水线的一系列痛点问题:

一、运维成本高,使用效率底下。Jenkins中可以通过编写Jenkinsfile的方式来构建一条流水线,
但是编写Jenkinsfile需要开发人员有Groovy语言基础,普通开发和运维人员想要自己定义个性化的流水线的成本较高,使用效率底下。

二、流水线层级划分粒度较粗。Jenkins中的一条流水线包括阶段(stage)和步骤(step)两个层级,一个stage中可以包含多个step,
但一个stage中的一部分step往往在处理同一个大的业务逻辑。这种层级的划分粒度较粗,不能很好地划分流水线内不同步骤之间的逻辑边界。

三、功能不够丰富。Jenkins流水线在功能性上不够丰富,当流水线运行后,Jenkins只支持流水线层面的取消和重试

为了解决原有基于 Jenkins 所搭建CI/CD系统的种种问题,本系统将设计与实现一种容器化的CI/CD流水线调度系统:第一、功能丰富,支持跳过、取消和重试等Jenkins不具备的功能;第二、灵活性好,流水线可自由编辑和编排;第三、性能优异,稳定性好。新的CI/CD平台可定制化的满足公司内部不同开发和测试人员的需求,集成内部成熟的基础工具服务,在业务高峰期提供可靠且稳定的服务,高效协同不同角色人员参与软件开发过程,缩短软件版本迭代周期,提高团队开发的效率。


\section{本文的主要工作}

\section{论文的组织结构}
第一章中最主要的是交代清楚为什么要做这个工作,根据前面调研内容才能
得出自己的工作究竟特色在哪?立足于解决以前产品所没有、或者不足的地方。
问题或具有挑战性的问题、创新点等一定要落到实处,不可含糊其辞。参考文献
的标注也主要集中在这部分。


\subsection{二级节标题}

\subsubsection{三级节标题}

\paragraph{四级节标题}

\subparagraph{五级节标题}
