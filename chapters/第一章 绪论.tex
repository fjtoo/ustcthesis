% !TeX root = ../main.tex

\chapter{绪论}

\section{系统开发背景和意义}
随着软件行业的快速发展,软件产品逐渐产生了需求变更频繁、产品迭代加快的趋势,这对软件的开发模式提出了新的挑战。
在过去,传统的软件开发团队普遍遵循瀑布模型,这一模型强调阶段划分,各个阶段必须严格按照时间先后进行,而且每个阶段都必须有产出物才能开始下个阶段\cite{瀑布模型}。
但是由于瀑布模型需要详尽的文档,这就要求在软件开发之初就对用户需求有着全面的了解,这在当下需求频繁变更的互联网时代几乎不可能;
同时,由于瀑布模型的集成式、阶段式的特点,产品的更迭速度势必缓慢,这也不符合现代的快速产品更迭的需求。
所以在当下,瀑布模型已经不符合大部分企业的要求,于是敏捷开发、DevOps、持续集成、持续交付等概念应运而生。

敏捷开发模型强调一组开发流程的反复迭代,要求开发人员采用XP、SCRUM等方法,通过紧密协作实现快速开发和交付。
敏捷开发鼓励以开发人员之间的频繁交流取代严格的文档,同时强调快速发布出软件的初代原型,以满足用户的基本需求,其余的功能和优化点将在后续的反复迭代中逐步完善\cite{敏捷开发}。
敏捷开发比瀑布模型更为灵活,但同时在开发人员和运维人员之间产生了矛盾:
开发人员追求快速迭代交付,而频繁的版本交付势必会加剧软件风险,与运维人员追求系统稳定形成了矛盾。

在开发与运维人员矛盾不断增加的背景下,DevOps(Development和Operations)理念产生并开始成为主流。
DevOps理念强调开发与运维的联合,旨在促进开发和运维过程中信息的流通,打破不同团队间的隔阂,以应对软件的快速变化\cite{DevOps}。
它要求运维人员不应等到软件交付后再进行运维,而是参与进软件的开发初期阶段,与开发人员的迭代周期进行同步的集成与交付。
但是,高频的测试、部署与交付显然不能依赖人力进行操作,这就要求DevOps拥有一套能够持续集成和持续交付的自动化工具。

持续集成(Continuous Integration,CI)和持续交付(Continuous Delivery,CD)是DevOps理念中重要的方法论。
持续集成是软件开发的迭代过程,团队成员不断上传代码至代码仓库,进行集成和测试,实现软件的快速迭代\cite{绪论持续集成1}。
这种方法不仅实现了DevOps的理念,及早发现问题和在早期交付可用原型,还通过强化开发团队与运维团队的联系,降低了团队成员的人工任务。
团队成员在持续集成中频繁集成工作成果,每日至少一次,通过自动构建的检验迅速发现集成错误。
与此同时,持续交付的核心思想是通过创建可重复、可靠的操作过程(一般称为流水线),将软件从概念变为现实\cite{绪论持续集成2}。
持续交付流水线将软件交付过程分成多个阶段,每个阶段验证新特性的质量,确认新功能,防止潜在错误对用户造成不良影响。
持续集成与持续交付继承了DevOps开发理念,成为现代软件工程中的必要环节,这种综合应用不仅简化了开发流程,也提高了软件开发的效率和质量。

当前,各个企业内均普遍使用持续集成与持续交付流水线(下文称“CI/CD流水线”)服务来完成内部的系统集成与交付\cite{1013369056.nh}。
市面上以Jenkins为代表的CI/CD流水线服务可以满足企业的基本开发需求,但随着企业业务的拓展与系统数量的增加,已经越来越难以满足不同团队的个性化需求。

结合以上背景,设计并且研发一套能够满足不同类型开发团队需求,且能保证高易用性、高稳定性的CI/CD流水线系统是非常有必要的。


\section{国内外研究现状}
CI/CD的概念,最早在 1990 年由 Grady Booch 于《Object oriented design with applications》一书中提出\cite{booch1990object},然后在 2001 年由 ThoughtWorks 公司的科学家 Martin Fowler 发表了题为《Continuous Integration》的文章,对CI/CD进行了系统性的阐述及推广\cite{CI首作},
同时 ThoughtWorks 公司也开发第一个真正意义的持续集成工具:Cruise Control\cite{绪论持续集成1},自此以后,企业和软件团队逐渐意识到了持续集成的价值和必要性。
在2006年,Martin Fowler再次发表了《Continuous Integration》新版本,对CI/CD工程化实践做了更完善的总结\cite{fowler2006continuous},此后越来越多的企业和软件团队开始尝试持续集成。

随着CI/CD概念与技术的日益普及,许多相关的工具被相继开发,如Circle CI、Travis CI、GitLab CI和Jenkins等。
Cruise Control以其多功能性著称,支持插件化的代码管理和通过邮件反馈集成结果\cite{徐仕成2007基于}。
Circle CI和Travis CI则尤其在GitHub社区中受到欢迎,
Circle CI提供自动化的软件构建和交付,同时提供灵活的启动选项和跨平台支持,包含Linux、MacOS、Android和Windows等系统\cite{hoang2020jamstack},
而Travis CI基于GitHub API,要求项目必须托管在GitHub上以使用其服务,为开源项目提供免费服务,对私有项目则是付费的\cite{2018Use}。
这两个工具的一个限制是它们不支持本地私有部署,不太适合在大型的软件团队和企业中使用。
与此相反,GitLab CI便提供了对GitLab托管代码仓库的直接集成,支持私有化部署。
作为GitLab的一个内置功能,GitLab CI为那些已经使用GitLab作为代码管理工具的开发团队提供便捷性,减少了初期配置的需要。
然而,GitLab CI的功能生态尚未完全成熟,可能不完全满足大型开发团队的所有需求。
Jenkins则作为一个开源的、支持广泛插件的工具\cite{林新党2014基于},不仅支持私有部署,还拥有易于安装和配置的特点。
其成熟的生态和丰富的插件库使得它被许多企业广泛采用。

国外的大型互联网企业也相继研发CI/CD平台,如Microsoft 的 Azure DevOps Pipelines,Amazon 的 Codepipeline,
Netflix 的 Spinnaker等,而且这些平台中一部分已经不仅仅是一个CI/CD平台,
而是成为了即开即用的 DevOps 平台,集成了从需求到开发,从测试到发布,从运维到运营的一站式企业协同研发,使得用户快速而又轻松地进行开发。


结合对开发CI/CD系统的实践,CI/CD相关研究也在同步进行。
2014年,Meyer 及其同事探讨了CI/CD的理念与工具,识别出构建持续集成系统的关键要素主要是版本控制系统和持续集成服务器\cite{Meyer2014Continuous}。
2015年,Lai 与 Leu 进行了深入探究,提出了针对 Web 应用的CI流程模型(WACIP),旨在加速Web应用的部署速度,减少开发风险\cite{2016Applying}。
Seth 与 Khare 分析了开源工具 Jenkins 在CI/CD中的应用,特别是如何合理利用 Jenkins 插件与不同的开发环境协同工作\cite{7453279}。
Rai 等研究人员则评估了 Jenkins 在集成开发环境中的性能,通过与其他集成工具的比较,展现了Jenkins 的高效与实用性\cite{2015A}。
Armenise 等人介绍了Jenkins 在新的设计趋势中的作用,强调其在软件开发周期中协调各参与团队的能力,确保软件按时交付\cite{2015Continuous}。


国内的CI/CD系统起步较晚,但发展很快,主要由一些大型互联网企业和云服务商打造,以满足企业内部软件团队的需要,提高企业研发效能,同时也向外部用户提供服务。
如腾讯的蓝盾平台、华为的DevCloud平台、阿里巴巴的云效平台等等。
这些平台往往基于企业自身在互联网行业的一系列成功实践,以Web的形式为用户提供可视化的友好界面,对内承载企业自有业务的CI/CD任务,对外也可以满足不同类型客户的多元化需求。

与此同时,容器技术以及容器集群管理技术的高速发展,为CI/CD的自动化部署提供了很好的载体\cite{docker},也使得容器化成为CI/CD系统的主要特征。
Docker作为轻量级的容器解决方案,它可以快速启动,快速部署,容易移植,天生就适合做持续交付与持续部署。
而谷歌的Kubernetes作为大规模容器集群编排管理引擎,支持自动化部署、管理大规模可伸缩的容器集群,也促进了CI/CD的发展。
在工业界,各大云服务商以及公司逐渐选择Kubernetes与Docker作为构建CI/CD平台的技术体系。

现如今,CI/CD流水线平台在国内外得到了极为广泛的应用。据调查,有六成的企业和软件团队使用Jenkins作为CI/CD平台\cite{DevOps中国调查研究};另一部分则自行搭建自有持续交付平台,而大型公司则更多的
侧重自研CI/CD平台,一方面是因为大型公司的需求更为复杂,业界所提供的云平台不能够满足他们的需求,另一方面是自研持续交付平台可以与自有基础工具等系统实现联动,实现更高效的开发,其他则是考虑到数据安全性等原因。然
而,CI/CD平台在真实场景下的落地实践中却困难重重,第一,CI/CD平台的建设涉及不同的人员角色,如开发人员、运维人员、质量保证人员,如何让不同角色人员进行良好的协同工作是建设过程中需要考虑的问题,第二,平台的建
设涉及多种基础条件,如自动化测试条件、测试环境隔离系统的成熟度等等\cite{rangnau2020continuous},第三,平台需要改变用户习惯,如何让用户平滑的过渡到新的持续交付平台也是建设过程中亟待解决的问题。
但是这些问题却阻止不了CI/CD平台的建设脚步,因为CI/CD平台对于开发团队所带来的效能提升是有目共睹的。

\section{解决的主要问题}
本系统致力于解决现有以Jenkins为代表的主流CI/CD流水线系统的一系列痛点问题:

一、运维成本高,使用效率底下。Jenkins中可以通过编写Jenkinsfile的方式来构建一条流水线,
但是编写Jenkinsfile需要开发人员有Groovy语言基础,普通开发和运维人员想要自己定义个性化的流水线的成本较高,使用效率底下。

二、流水线层级划分粒度较粗。Jenkins中的一条流水线包括阶段(stage)和步骤(step)两个层级,一个stage中可以包含多个step,
但一个stage中的一部分step往往在处理同一个大的业务逻辑。这种层级的划分粒度较粗,不能很好地划分流水线内不同步骤之间的逻辑边界。

三、功能不够丰富。Jenkins流水线在功能性上不够丰富,当流水线运行后,Jenkins只支持流水线层面的取消和重试,无法满足用户在流水线执行过程中多样化的人工干预需求。


\section{本文的主要工作}
本文分析了原有以 Jenkins 为代表的主流CI/CD系统的局限,以此为基础设计并实现了一种容器化的CI/CD流水线调度系统,有以下特点:第一、功能丰富,支持对流水线中各个层级的跳过、取消和重试等Jenkins不具备的功能;
第二、灵活性好,流水线可自由编辑和编排;
第三、性能优异,稳定性好。从功能方面,系统包含了流水线管理、镜像管理和节点管理三个功能模块,提供了从配置到实际运行环境的CI/CD流水线全流程编排与执行。
从设计方面,系统在架构上将整个系统拆分为Backend、调度器和执行器三个架构模块,使得整个系统模块功能明确,不同模块之间的耦合度低且非常易于扩展。

本文将从开发团队、测试团队和运维团队三种不同的利益相关者的需求出发,详细阐述系统的系统的概要设计与具体实现,并对系统进行功能测试与性能测试,
为高可用、高易用性的CI/CD平台建设提供一种可行的解决方案。

\section{论文的组织结构}
本文一共有七个章节,结构如下:

第一章为绪论,开篇介绍了CI/CD流水线调度系统的开发背景,然后对国内外CI/CD相关理论研究和系统开发进行介绍,梳理出本文的主要工作和解决的主要问题,最后给出了论文的组织结构。

第二章为相关技术概括,精简地介绍了系统中涉及以及使用的各项关键技术,包括CI/CD流水线、Docker容器化技术、容器集群调度引擎Kubernetes、GitLab Runner和消息队列RocketMQ,
这些技术将设计与实现CI/CD流水线调度系统的过程中被使用。

第三章为需求分析,首先进行了系统概述,接下来从开发团队、测试团队和运维团队三种不同的利益相关者的角度识别需求,进行了初步的需求导出。
然后阐述了系统的功能性需求,以用例为主要载体分析了系统中应当实现的各种功能。最后进行了非功能性需求分析,从性能、稳定性等角度给出了系统需要满足的非功能性需求。

第四章为系统概要设计,这一部分主要描述CI/CD流水线调度系统的总体框架,主要关注于系统的主要部件与部件间的连接。
首先进行了系统总体设计,分解出了系统架构。然后分别从模块设计的角度,将系统分为了Backend模块、调度器模块、执行器模块和消息队列,并分别介绍其设计思想和架构设计。
接下来根据功能模块将系统划分为流水线管理、节点管理和镜像管理三个模块,并分析了各个模块中重点功能的设计思路。最后进行了数据模型设计,分析了系统中数据表与实体之间的关系。

第五章为系统详细设计与实现,这一部分依据系统概要设计中的划分方式,将系统中的重点功能和模块通过类图、状态图和时序图等方式,详细阐述了功能与模块的内部实现逻辑。

第六章为系统测试,首先介绍了测试环境的配置,然后分别为流水线管理、人工干预、镜像管理和节点管理四个模块设计了详尽的测试用例。
同时进行了性能测试,对Backend内接口的响应时间和调度器与执行器的响应时长进行了测试,并最终得出测试结论。

第七章为结论与展望,首先总结了本文的内容,最后分析了系统的不足以及未来可以改进的方向。